\documentclass{beamer}
%
% Choose how your presentation looks.
%
% For more themes, color themes and font themes, see:
% http://deic.uab.es/~iblanes/beamer_gallery/index_by_theme.html
%
\mode<presentation>
{
  \usetheme{default}      % or try Darmstadt, Madrid, Warsaw, ...
  \usecolortheme{default} % or try albatross, beaver, crane, ...
  \usefonttheme{default}  % or try serif, structurebold, ...
  \setbeamertemplate{navigation symbols}{}
  \setbeamertemplate{caption}[numbered]
} 

\usepackage[english]{babel}
\usepackage[utf8x]{inputenc}
\usepackage[font=small]{caption}
\graphicspath{ {./img/} }

\title[PR1/2 Demo]{PR1/2 Demo}
\author{Christian Permann}
\institute{Faculty of Computer Science, University of Vienna,\newline W\"ahringer Stra{\ss}e 29, 1090 Vienna}
\date{20.01.2019}

\begin{document}

\begin{frame}
  \titlepage
\end{frame}

% Uncomment these lines for an automatically generated outline.
%\begin{frame}{Outline}
%  \tableofcontents
%\end{frame}

\section{Introduction}

\begin{frame}{Topic}

Finding parameters and good solutions in the domain of clustering is difficult. Different solutions may be equally valid depending on the questions asked and the different data sets. There exist some quality measures but those do not take the intrinsic meaning of the result into account.

\vspace{5mm}
\textbf{Proposed solution:}

Perform Meta-Clustering and let the user explore the different results comparatively via a visualization.

\end{frame}

\section{Tool}

\begin{frame}{The Tool}

\centerline{Demo}

\end{frame}


%\bibliography{citations}
\bibliographystyle{ieeetr}

\end{document}